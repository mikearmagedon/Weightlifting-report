The fourth step on the ontology design and development phase aims the identification of the developed properties for the ontology. These developed properties allow the establishment of relationships between one or more concepts.

Regarding the Weightlifting ontology, two types of OWL properties were developed: \textbf{Object} and \textbf{Data} properties. 

Object properties were developed to define explicit relationships between one or more concepts. Such properties as 'practice', 'lifts', 'has', 'isPracticedby', 'isliftedBy' and 'isPartOf', make possible the establishment of relations between one or more concepts that compose the target ontology. As already said, each property establishes relationships and their descriptions are presented below: 

\begin{itemize}  
	\item Object property 'practices' relates Athlete class to Exercise class and assures that the Athlete always practices an exercise.
	\item Object property 'lifts' relates Athlete class and Barbell class, meaning that an athlete lifts a barbell.
	\item Object property 'has' relates Exercise class and Barbell class, meaning that an weightlifting exercise has a barbell.
	\item Object property 'isPracticedby' is the inverse property of 'practices'.
	\item Object property 'isliftedBy' is the inverse property of 'lifts'.
	\item Object property 'isPartOf' is the inverse property of 'has'.
\end{itemize}

Data properties were developed to define explicit relationships between concepts and data values.
These properties were developed to respond to some cases where three-dimensional variables are required for the measurement of specific lifting positions, relating to the athlete during exercise.
Such properties as the three-dimensional positions of the knees, hips, ankles, shoulders, elbows, heels and bar along the athlete's lifting series are example of properties developed to enable the establishment of relationships between objects and data values.