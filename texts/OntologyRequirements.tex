Regarding the project requirements, the vast majority of them referred to the application instead of the ontology. The presented functional requirements for the application concern mainly the user interaction and the non-functional ones described the response time and friendliness of the application. These requirements have been met when the project was over.

Concerning the presented ontology requirements, the first one detailed how the system must apply the rules and query the ontology, and the second one assumed that the system should give some recommendations to the user, from the rules inferred. Both of these requirements were almost fulfilled given the ontology limitations. One of the main issues found was the lack of a range of accepted values in the rules that infer if the move was correct. This solution would have improved the system reliability on the results obtained.

The ontology structure developed is very minimalistic, which allows its integration in different projects within the area of weightlifting but not without any modifications. The ontology simplistic nature is also its downside, failing to detail some parts of its domain like muscles or specific parts of the body.